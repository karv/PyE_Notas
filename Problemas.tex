\documentclass{report}
\usepackage[utf8]{inputenc}

\begin{document}
	
\section*{Problemas de combinatoria}
\begin{enumerate}
\item En el lenguaje de computadoras, la información está descrita con 1's y 0's. Un byte es una secuencia de este tipo de longitud 8.
¿Cuántos diferentes bytes existen?

\item ¿Cuántos números capicúa existen que son menores que $10^6$?

\item Se quieren repartir cuatro dulces diferentes entre dos niños. ¿De cuántas formas se puede hacer si a cada niño de tener dos dulces? Liste todas las combinaciones posibles.

\item ¿De cuántas formas se pueden sentar 10 personas en 15 asientos?

\item En un torneo de fútbol en el que participan 20 equipos, cada equipo juega exactamente una vez contra cada uno del resto de los equipos. ¿De cuántos partidos consta en torneo?

\item ¿De cuántas formas se puede ordenar un mazo de 52 cartas de póquer?

\item Se realizó un concurso de música en el que participaron 28 competidores. ¿De cuántas formas distintas se puede seleccionar un primer, un segundo y un tercer lugar?

\item Se reparten las 28 fichas de un dominó entre cuatro jugadores. ¿De cuántas formas se puede hacer esta repartición?

\item En una retícula rectangular de 5 por 4, ¿De cuántas formas se puede llegar desde la esquina superior izquierda a la inferior derecha, si los únicos movimientos permitidos son moverse a la derecha y hacia abajo?

\item ¿Cuántas diagonales tiene un polígono de $n$ lados?

\item Considerando que el año tiene 365 días, ¿cuál es la probabilidad de que dos personas tengan la misma fecha de cumpleaños?

\item ¿Cuál es la probabilidad de que tres personas tengan fechas de cumpleaños todas distintas entre sí?

\item Para formar un equipo de baloncesto se necesitan escoger a 5 jugadores de 10 disponibles, y de los elegidos escoger al capitán. ¿De cuántas formas se puede hacer esto?

\end{enumerate}
\end{document}